\chapter{Cuerpo de la tesis}
\label{cap:cuerpo}

\section{Primera parte}
\label{sec:primera}

Lorem ipsum dolor sit cuchufl\'i barquillo bac\'an jote gamba listeilor po cahu\'in, luca mel\'on con vino pichanga coscacho ni ah\'i peinar la muñeca chuchada al chancho achoclonar. Chorrocientos pituto ubicatex huevo duro bolsero cachureo el hoyo del queque en cana huev\'on el año del loly hacerla corta impeque de miedo quilterry la raja longi ñecla. Hilo curado rayuela carrete quina guagua lorea piola ni ah\'i con la Figura \ref{fig:ejemplo}.

\begin{figure}[!ht]
	\centering
	\captionsetup{justification=centering}
	\includegraphics[scale=0.6]{images/Ejemplo.png}
	\caption[Ejemplo de figura.]{Ejemplo de figura.\\Fuente: Elaboraci\'on propia}
	\label{fig:ejemplo}
\end{figure}

\section{Segunda parte}
\label{sec:segunda}

Lorem ipsum dolor sit cuchufl\'i barquillo bac\'an jote gamba listeilor po cahu\'in, luca mel\'on con vino pichanga coscacho ni ah\'i peinar la muñeca chuchada al chancho achoclonar. Chorrocientos pituto ubicatex huevo duro bolsero cachureo el hoyo del queque en cana huev\'on el año del loly hacerla corta impeque de miedo quilterry la raja longi ñecla. Hilo curado rayuela carrete quina guagua lorea piola ni ah\'i con la Tabla \ref{tab:ejemplo}.

\begin{table}[!ht]
	\centering
	\caption{Tabla de ejemplo.}
	\begin{tabular}{| c | c |}
		\hline
		A & B \\ \hline
		1 & 4 \\
		2 & 5 \\
		3 & 6 \\\hline
	\end{tabular}
	\label{tab:ejemplo}
\end{table}

\section{Tercera parte}
\label{sec:tercera}

Lorem ipsum dolor sit cuchufl\'i barquillo bac\'an jote gamba listeilor po cahu\'in, luca mel\'on con vino pichanga coscacho ni ah\'i peinar la muñeca chuchada al chancho achoclonar. Chorrocientos pituto ubicatex huevo duro bolsero cachureo el hoyo del queque en cana huev\'on el año del loly hacerla corta impeque de miedo quilterry la raja longi ñecla. Hilo curado rayuela carrete quina guagua lorea piola ni ah\'i con el Algoritmo \ref{alg:ejemplo}.

\begin{algorithm}[!ht]
	\caption{Algoritmo de ejemplo.}
	\label{alg:ejemplo}
	\begin{algorithmic}[1]
	\REQUIRE Entrada de ejemplo.
	\ENSURE Salida de ejemplo.	
	
	\IF {esto est\'a bien}
		\STATE hacer algo
	\ELSE
		\STATE hacer otra cosa
	\ENDIF
	
	\RETURN Retornar ejemplo
	
	\end{algorithmic}
\end{algorithm}